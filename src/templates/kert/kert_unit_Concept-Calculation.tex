


% ====================================================================
\chapter{Introduction}
\label{sec:Introduction}

% --------------------------------------------------------------------
\section{Purpose}
\label{sec:Purpose}

Describe the purpose for this algorithm.

% ====================================================================
\chapter{Algorithm Details}
\label{sec:AlgorithmDetails}

% --------------------------------------------------------------------
\section{Algorithm Overview}
\label{sec:AlgorithmOverview}

Include an overview of the algorithm from a black box level.
Describe how the algorithm is related to other algorithms.

% --------------------------------------------------------------------
\subsection{Algorithm Inputs}
\label{sec:AlgorithmInputs}

Describe any inputs that the algorithm uses to make
calculations from. Inputs are summarized in \autoref{tab:CCddd-input}.

\begin{table}
  \centering
  \caption{Input information for CCddd.}
  \label{tab:CCddd-input}
  \begin{CCInOutTable}
    % Five (5) columns
    % symbol & description & unit & range & notes \\
    $m$ % mathematical symbol
    & mass % short description
    & \si{\kilo\gram} % dimension of quantity
    & $\mathbb{R} \in (-\infty, \infty)$ % math-valid range
    & {}\footnote{Technically this is the \emph{intertial} mass as opposed to gravitational mass.}\footnote{Negative mass is allowed by the mathematics so it is not an error. However, since negative inertial mass often yields physical paradoxes it is worth a warning.} % notes
    \\%
    $\vec{a}$ % mathematical symbol
    & acceleration\footnote{One-dimensional vector, direction indicated by sign.} % short description
    & \si{\meter\per\second\squared} % dimension of quantity
    & $\mathbb{R} \in (-\infty, \infty)$ % math-valid range
    \\%
  \end{CCInOutTable}
\end{table}


% --------------------------------------------------------------------
\subsection{Algorithm Outputs}
\label{sec:AlgorithmOutputs}

Describe the outputs from the algorithm. Outputs are summarized in \autoref{tab:CCddd-output}.

\begin{table}
  \centering
  \caption{Output information for CCddd.}
  \label{tab:CCddd-output}
  \begin{CCInOutTable}
    % Five (5) columns
    % symbol & description & unit & range & notes \\
    $\vec{F}$ % mathematical symbol
    & force % short description
    & \si{\newton} % dimension of quantity
    & $\mathbb{R} \in (-\infty, \infty)$ % math-valid range
    & % notes
    \\%
  \end{CCInOutTable}
\end{table}

% --------------------------------------------------------------------
\section{Algorithm Description}
\label{sec:AlgorithmDescription}

Detailed description of the algorithm with sufficient fidelity
to allow independent implementation.

% --------------------------------------------------------------------
\subsection{Assumptions / Limitations}
\label{sec:AssumptionsLimitations}

Describe any known limitations or assumptions that went into
the design.  

%-%% ====================================================================
%-%\chapter{Trade Studies}
%-%\label{sec:TradeStudies}
%-%
%-%Provide details and results of any trade studies used in the
%-%design of this algorithm.  Should include any optimization studies
%-%used to determine parameter values.  This may just be a summary with
%-%reference to other documents.

% ====================================================================
\chapter{Algorithm Verification}
\label{sec:AlgorithmVerification}

Provide test cases and associated metrics / tolerances to be
used to test that any implementation of the algorithm is correct. Refer to \autoref{tab:CCddd-verification} for a summary of tests.

\begin{table}
  \centering
  \caption{Verification test cases for CCddd.}
  \label{tab:CCddd-verification}
  % These lengths are necessary to avoid white space from the \{top,mid,bottom}rule lines
  \setlength{\aboverulesep}{0pt}
  \setlength{\belowrulesep}{0pt}
  \setlength{\extrarowheight}{.75ex}
  \begin{tabular}{lll>{\columncolor{green}}l}
    \toprule%
    \multicolumn{1}{c}{\bfseries ID}%
    &\multicolumn{1}{c}{\bfseries $m$}%
    &\multicolumn{1}{c}{\bfseries $\vec{a}$}%
    &\multicolumn{1}{>{\columncolor{green}}c}{\bfseries $\vec{F}$}%
    \\%
    \midrule%
    1 & 1 & 1 & 1%
    \\%
    2 & 3.300 & -9.800 & -32.34%
    \\%
    \bottomrule
  \end{tabular}
\end{table}


















